\documentclass[12pt,a4paper,ngerman]{scrartcl}

% Dateikodierung ist latin1%
\usepackage[utf8]{inputenc}
\usepackage[ngerman]{babel}
\usepackage{listings}
\usepackage[usenames,dvipsnames,svgnames,table]{xcolor}
\usepackage{verbatim} 
\usepackage{soul}
\usepackage[normalem]{ulem}
\usepackage{sectsty}
\usepackage{varwidth}
\usepackage{truncate}
\usepackage{hyperref}
\usepackage{graphicx}
\usepackage{float}
\usepackage{framed}
\usepackage{mdframed}
\usepackage{array}
\usepackage{colortbl}

\definecolor{dkgreen}{rgb}{0,0.6,0}
\definecolor{gray}{rgb}{0.5,0.5,0.5}
\definecolor{mauve}{rgb}{0.58,0,0.82}

% Setting the properties for links, urls etc.
\hypersetup{
    colorlinks,
    citecolor=black,
    filecolor=black,
    linkcolor=black,
    urlcolor=blue
}

\lstset{ %
  language=C++,                     % the language of the code
  basicstyle=\footnotesize\ttfamily,    % the size of the fonts that are used for the code
  numbers=left,                     % where to put the line-numbers
  numberstyle=\tiny\color{gray},    % the style that is used for the line-numbers
  stepnumber=1,                     % the step between two line-numbers. If it's 1, each line will be numbered
  numbersep=5pt,                    % how far the line-numbers are from the code
  backgroundcolor=\color{white},    % choose the background color. You must add \usepackage{color}
  extendedchars=true            %
  showspaces=false,                 % show spaces adding particular underscores
  showstringspaces=false,           % underline spaces within strings
  showtabs=false,                   % show tabs within strings adding particular underscores
  frame=single,                     % adds a frame around the code
  rulecolor=\color{gray},           % if not set, the frame-color may be changed on line-breaks within not-black text (e.g. comments (green here))
  tabsize=4,                        % sets default tabsize to 2 spaces
  captionpos=b,                     % sets the caption-position to bottom
  breaklines=true,                  % sets automatic line breaking
  breakatwhitespace=false,          % sets if automatic breaks should only happen at whitespace
  title=\lstname,                   % show the filename of files included with \lstinputlisting; also try caption instead of title
  keywordstyle=\color{blue},        % keyword style
  commentstyle=\color{dkgreen},     % comment style
  stringstyle=\color{mauve},        % string literal style
  escapeinside={\%*}{*)},           % if you want to add LaTeX within your code
  morekeywords={*,...}          % if you want to add more keywords to the set
}

% Color definitions
\definecolor{crucial}{RGB}{0,60,0} % This color marks the portions of code that are crucial (see below)
\definecolor{cream}{RGB}{255,255,204} % highlighting color
\definecolor{blue}{RGB}{0,0,255}

\newcommand{\hlc}[2][cream]{{\sethlcolor{#1} \hl{#2}}}

\subsectionfont{\normalsize} % We change the sections to have a slightly smaller size

% Crucial marks keywords or statements that are crucial for passing the exam ;-)
\newcommand{\crucial}[1]{\hlc[cream]{#1}} % Optimized for online reading
%\newcommand{\crucial}[1]{\textbf{\textcolor{crucial}{\uline{#1}}}} % custom
%\newcommand{\crucial}[1]{\uline{#1}} % Optimized for print

% Question Command
\newcommand{\question}[1]{
	\subsection[\truncate{0.80\textwidth}{#1}]{
	\textcolor{blue}{#1}}}

\newenvironment{multilinequestion}[1][]
	{\subsection[\truncate{0.80\textwidth}{#1}]{\textcolor{blue}{#1}} \color{blue}}
	{}

% Used for the right vertical bar next to the answer blocks
\newlength\RightBarWidth
\setlength\RightBarWidth{1pt}
\newenvironment{rightbar}
	{
	\def\FrameCommand##1{%
	\hspace{\textwidth}\hspace{-\RightBarWidth}%
	{\color{dkgreen}\vrule width \RightBarWidth}
	\hspace{-\textwidth}##1\hfill}%
	\MakeFramed{\addtolength\hsize{-\width-\RightBarWidth-\columnsep}%
	\FrameRestore}%
	}
	{\endMakeFramed}

% Make footnotes available inside frames
\makeatletter

\newcommand\mystuff@footnotebuffer{}
\newcounter{mystuff@footnote}

\newcommand\bufferfootnotes{
	\let\mystuff@footnoteold\footnote
	\setcounter{mystuff@footnote}{\thefootnote}
	\renewcommand\mystuff@footnotebuffer{}
	\renewcommand{\footnote}[1]{
		\footnotemark
		\g@addto@macro{\mystuff@footnotebuffer}{
			\stepcounter{mystuff@footnote}
			\protect\footnotetext[\themystuff@footnote]{##1}
		}
	}
}

\newcommand\stopbufferingfootnotes{%
	\mystuff@footnotebuffer%
	\renewcommand{\footnote}[1]{\mystuff@footnoteold{##1}}%
}

\makeatother

% Answer environment
\newenvironment {answer}
                {\bufferfootnotes\begin{rightbar} }
                {\end{rightbar}\stopbufferingfootnotes}
                  
\title{Betriebssyteme 1 \\ Fragenkatalog \\ WiSe 2012/2013 \\[5pt] \Large{Universität Bremen}}
\date{\today}

\begin{document}

% Insert title here
\maketitle

% Abstract section
\begin{abstract}
Die hier zusammengetragenen Fragen und Antworten dienen der vorbereitung auf eine Modulprüfung bzw. Fachgespräch im Fach Betriebssysteme 1. Jeder kann dazu beitragen die Antworten zu erweitern, zu verbessern oder zu aktualisieren, indem er die Repository auf Github clont, das Dokument überarbeitet und anschließend Merge-Requests einreicht. Das Repository befindet sich unter  \url{https://github.com/momesana/bs1-fragenkatalog/}.
\end{abstract}

\newpage

% This inserts the table of contents
\tableofcontents 

% Templates for question and answer pairs
%
% Simple question
%\question{}
%\begin{answer}
%\end{answer}
%
% Multilone question: To be used if the question contains images or blocks etc.
%\begin{multilinequestion}[text to appear in the subsection]
%\end{multilinequestion}
%the question that spans many lines
%\begin{answer}
%the answer
%\end{answer}

% This included Fragment contains questions regarding Operating Systems
\section{Grundlagen}




\section{Interprozesskommunikation}

M"oglichkeiten der Interprozesskommunikation kann "uber gemeinsamen Speicher erfolgen. 
\begin{enumerate}
	\item Globale Variablen f"ur Threads
	\item gemeinsam genutze Dateien
	\item Shared Memory
	\item User Register
	\item Umgebungsvariablen
\end{enumerate} 
Darüber hinaus ist es m"oglich, Nachrichten an andere Programme zu schicken.
\begin{enumerate}
	\item Signale
	\item Sockets
	\item Pipes
\end{enumerate}

Ein Ziel bei der Interprozesskommunikation ist die Serialisierbarkeit. Wenn zwei oder mehr Prozesse durch konkurierende Operationen auf den selben Speicher zugreifen, kann es zu Race Conditions kommen. Konkurierende Operationen hei"sen serialisierbar, wenn eine sequenzielle Ausführung existiert, welche den gemeinsamen Speicher im selben Nachzustand hinterl"asst.
\\\\
Def. Race Condition:
Zwei oder mehr Prozesse (oder Threads) greifen auf gemeinsamen Speicher zu. Das Ergebnis ist nicht deterministisch.
\\\\
Def. Kritischer Abschnitt:
Critical Sections sind Codeabschnitte in meheren Prozessen, deren nebenläufige Ausführung die Serialisierbarkeit verletzen würde. Sie gehen vom laden der Daten bis hin zum Speichern. Der Kritische Abschnitt befindet sich direkt dazwischen. 
\\\\
Def. Gegenseitiger Auschluss:
Die ausführung kritischer Abschnitte schlicßt sich wechselseitig aus (mutual exclusion).
\\\\
Der Algorithmus von Peterson ist ein Serialisierungsverfahren f"ur zwei Prozesse. 
\lstset{language=c} 
\begin{lstlisting}[breaklines,showstringspaces=false,frame=none] 
// globale Variablen
int interested[2] = {0, 0};
int turn = 0;

void enterCritical(int pid) {
	interested[pid] = 1;
	turn = pid;

	while(turn == pid && interested[1-pid]);
}

void leaveCritical(int pid) {
	interested[pid] = 0;
} 

\end{lstlisting} 

Das Protokoll von Fischer gilt auh f"ur mehere Prozesse, jedoch kann es sein, dass ein Prozess niemals in den kritischen Abschnitt eintreten kann. Der Algorithmus ist nicht verhungerungsfrei.
\lstset{language=c} 
\begin{lstlisting}[breaklines,showstringspaces=false,frame=none] 
// globale Variablen
volatile int turn = 0;

void enterCritical(int pid) {	// pid >= 1
	do {
		while(turn != 0);
		turn = pid;

		sleep();
	} while(turn != pid);
}

void leaveCritical(int pid) {
	turn = 0;
} 

\end{lstlisting}

Ist ein anderer Prozess in einem kritischen Abschnitt, muss der gerade ausgeführte warten. Ein Verfahren dazu ist das Active Wait Verfahren (Spinlocks). \\
Vortiel: Keine Betriebssystem Beteiligung\\
Nachteil: wartender Prozess verbraucht Rechenzeit.
\\\\
Def. Priorit"ateninversion:
P1 hat eine gute Priorit"at, P2 hat keine gute. Der Prozess P2 befindet sich in dem kritischen Abschnitt. Der Scheduler entschlie"st sich, dem h"oher Priorisierten Prozess CPU Zeit zuzuteilen. Da P2 noch in der CS steckt, kann P1 nicht eintreten und vergeudet seine Zeit im Spinlock. 
\\\\
Def. Livelock:
Program Counter ver"andert sich noch, es k"onn jedoch keine Nutzdaten mehr verarbeitet werden.
\\\\
Def. kooperatives Multitasking:
Prozesswechsel erfolgt nur, wenn der Prozess freiwililg die CPU Zeit abgibt. 
\\\\
Def. Pr"aemptiver Scheduler:
k"onnen einem Prozess die CPU Zeit wegnehmen. Werden durch clock interrupts ausgeführt. 
\\\\
Der Reader Writer Algorithmus (Non blocking write Protokoll):
Annahme: nur ein Leser und ein Schreiber.
Writer beschreibt Datenblock beliebiger L"ange. Der Reader lie"st den Datenblock, verarbeitet jedoch nur die Daten, bei denen er vom Schreiber nicht unterbrochen wurde. Dieses verfahren bezeichnet man als optimistisches Verfahren. Der Schreiber kann immer Schreiben. Der Reader muss prüfen, ob zwischenzeitlich der Schreiber aktiv war, bevor er seine daten verabeiten kann.

\lstset{language=c} 
\begin{lstlisting}[breaklines,showstringspaces=false,frame=none] 
// globale Variablen
int ccF = 0; // Concurency Control Field, 

//Writer:
	ccF = ccF + 1;
	// write in buffer
	ccF = ccF + 1; 

// Reader:
	do {
		int ccF_begin, ccF_end;
		do {
			ccF_begin = ccF;
		} while(ccF_begin % 2); // nur lesen wenn gerade

		// Hier wird nun gelesen

		ccF_end = ccF;

	} while(ccF_begin != ccF_end);

\end{lstlisting}

\section{Nebenläufigkeit}

    
\subsection{Race conditions}
Ein \textbf{race condition} ist in der Programmierung eine Konstellation, in der das \crucial{Ergebnis einer Operation vom zeitlichen Verhalten (bzw. Ausf"uhrungsreihenfolge) bestimmter Einzeloperationen abh"angt}. Das Ergebnis wäre also nichtdeterministisch.

\subsection{Synchronisation}
Die Synchronisation dient dem \crucial{Zweck} \crucial{racing conditions} zu \crucial{vermeiden}. Das \crucial{Ziel} der Synchronisation ist die \crucial{Serialisierbarkeit!}

\subsection{Serialisierbarkeit}
\crucial{Reihenfolge bei konkurrierendem Zugriff entspricht seriellem Zugriff}.

\paragraph*{}
Beispiel: Prozesse $P_1, ..., P_n$ operieren auf eine Datenstruktur $d$. Die konkurrierende Operationen auf $d$ heißt serialisierbar, wenn es eine sequentielle Ausführung $P_{i_{1}}, P_{i_{2}}, ..., P_{i_{n}}$ gibt, die $d$ im selben Nachzustand hinterlässt.

\subsection{Critical Section}
\crucial{Code-Abschnitte in Prozesse $P_1, ..., P_2$ deren nebenl"aufige Ausf"uhrung die Serialisierbarkeit verletzen w"urde}.

\subsection{Mutual Exclusion}
\crucial{Die Ausf"uhrung im critical section schlie"st sich Wechselseitig aus}.

\subsection{Peterson's Algorithmus}

Aufruf durch einen Prozess:
\begin{lstlisting}
enter(pid); // pid is 0 or 1
// process CS...
leave(pid);
\end{lstlisting}


\begin{lstlisting}
// global
int turn; // 0 or 1
int interested[] = {0, 0}

void enter(int pid) { // pid is either 0 or 1
	interested[pid] = 1;
	turn = pid;
	while (turn == pid && interested[1-pid]); // active wait
}

void leave(int pid) {
	interested[pid] = 0;
}
\end{lstlisting}


\subsection{Strict Alternation}
Wandelt man Peterson's Algorithmus leicht ab, indem man sich der \texttt{interested} variable entledigt ist das Resultat \textbf{strict alternation}. Dabei wartet ein Prozess so lange, bis ein anderer Prozess versucht in die critical section einzutreten. Erst dann kann er die kritische Sektion betreten.

\begin{lstlisting}
// global
int turn; // 0 or 1

void enter_critical(int pid) { // pid is either 0 or 1
	turn = pid;
	while (turn == pid); // active wait
}
\end{lstlisting}


\subsection{Fischers Protokoll}
\begin{lstlisting}
// globale Variablen 
volatile int turn = 0; 
void enterCritical(int pid) {	// pid >= 1 	
	do { 		
		while(turn != 0); 		
		turn = pid; 	
	} while(turn != pid); 
} 

void leaveCritical(int pid) { 	turn = 0; } 
\end{lstlisting}

Nachteil: Ein Prozess könnte theoretisch verhungern.

\subsection{Verhungerungsfreiheit (Free of Starvation)}
Wenn ein Prozess kontinuierlich bereit ist, bekommt er schließlich die Ressource (critical section).

\subsection{Spinlocks (Active Wait)}
Aktives Warten (spinlock) ist das \crucial{Verharren} eines Prozesses \crucial{in einer Schleife, bis} der zu betretende \crucial{kritische Abschnitt} wieder \crucial{frei ist}.


% A table may be more appropriate
\begin{description}
\item[Vorteil:] \hfill \\
	Keine Betriebssystem-Beteiligung (kein teurer Kontextwechsel erforderlich)
\item[Nachteil:] \hfill \\
	Wartende Prozesse verbrauchen Rechenzeit
\end{description}

\begin{verbatim}
    Gemeinsam von A und B genutzte Variable: lock
    Interpretation des Werts:                0 gesperrt, ungleich 0 offen 
    Initialisierung:                         lock = 0
           
    Prozess A                    Prozess B
       ...                          ...
       solange (lock == 0) {        ...
          ;                         lock = 1; Aktion b
       }                            ...
       Aktion a                     ...
\end{verbatim}

\subsection{Prioriäten Inversion}
Szenario: P1 (gute Priorität), P2 (schlechte Priorität) werden auf einem core ausgeführt. P2 kommt in die CS und P1 muss aktiv warten.

\paragraph*{Wirkung:}
P1 bekommt mehr Rechenzeit, die aber in AW (aktive wait) vergeudet wird, da P2 vom Scheduler aufgrund seiner schlechten Priorität immer wieder unterbrochen wird und die CS noch nicht verlassen hat.

\subsection{Lifelock}
\crucial{Keine Nutzdatenverarbeitung obwohl sich der Programmcounter "andert}. Im gegensatz dazu stehen bei einem Deadlock alle Programmcounter fest.

\subsection{Kooperatives Multitasking und präemptives Scheduling}

\begin{description}
\item [Kooperatives Multitasking] \hfill \\
	Prozesswechsel erfolgt \ul{nur}, wenn der aktive Prozess \ul{freiwillig} die CPU wieder abgibt. (sogenanntes nicht-präemptives Scheduling)
\item [Präemptives Scheduling] \hfill \\
	Pr"aemptive Scheduler k"onnen einem aktiven Prozess die CPU \uline{entziehen} (Technik: Clockinterrupt führt zum Scheduler-Aufruf).
\end{description}


\subsection{Reader / Writer (nonblocking write protocol)}
Annahme: 1 leser (liest den Datenblock) und 1 schreiber (beschreibt den Datenblock) \\
Ziel: Der Reader verarbeitet nur konsistene Daten, bei denen er beim Lesen nicht durch den Writer unterbrochen wurde.

\subsubsection{Optimistisches Verfahren}
\begin{enumerate}
\item Writer darf immer schreiben
\item Reader liest \crucial{optimistisch}, prüft dann , ob die Daten konsistent sind
\end{enumerate}

\begin{lstlisting}
// global variables
int ccf = 0;
\end{lstlisting}

\begin{lstlisting}
// writer
ccf = ccf + 1;
// write data
ccf = ccf + 1; 
\end{lstlisting}

\begin{lstlisting}
// reader
do {
	int ccfbegin, ccfend;
	do {
		ccfbegin = ccf;
	} while (ccfbegin % 2);
	// read data
	ccfend = ccf;
} while (ccfbegin != ccfend);
\end{lstlisting}


\subsection{Ringpuffer}

\subsubsection{Eigenschaften}
\begin{itemize}
\item Reader-Writer Kommunikation mit einem Leser und einem Schreiber
\item FIFO-Buffer
\item Ohne Betriebssystem-Beteiligung
\item Ungelesene Daten dürfen \uline{noch} nicht überschrieben werden
\item ungeschriebene Daten dürfen \uline{noch} nicht gelesen werden
\item Puffer ist leer, wenn \texttt{rIdx == wIdx}
\item Puffer ist voll, wenn \verb|(wIdx + 1) % n == rIdx|
\item Bedingung fürs Lesen \texttt{(rIdx != wIdx)}
\item Bedingung fürs Schreiben \verb|(wIdx + 1) % n != rIdx|
\item Bei einer Arraygröße von $n$ ergibt sich eine Kapazität von $n -1$
\item \verb|%| (Modulo) Operation ist unter umst"anden teuer. W"ahle $n=2^k$ mit 
$k \in \mathbb{N}$ und ersetze \verb|%n| durch \verb|&(1<<k)-1|
\end{itemize}

\begin{lstlisting}
void write(t in) {
	unsigned nextWIdx = (wIdx + 1) % n;
	while (nextWIdx == rIdx); // spinlock
	buffer[wIdx] = in; // write
	wIdx = nextWIdx; // update write index
}

t read() {
	t out;
	while (rIdx == wIdx); // spinlock
	out = buffer[rIdx]; 
	rIdx = (rIdx + 1) % n;
	return out;
}
\end{lstlisting}

\subsection{Implementierung des Consumer Producer Konzeptes mit Semaphoren}

\subsubsection{Semaphoren}

Semaphoren dienen der Sicherung von kritischen Abschnitten. Sie verfügbt über zwei grundlegende Funktionen:

\begin{enumerate}
\item lock anfordern
\item lock freigeben
\end{enumerate}

\begin{itemize}
\item lock wird angefordert
\item wenn der lock nicht verfügbar ist, wird der Prozess schlafen gelegt (aus der Liste der aktiven Prozesse (runqueue) entfernt)
\item wenn lock bezogen wurde
\end{itemize}

\begin{description}
\item[semget] \hfill \\
	 Semaphore beantragen
\item[semctl] \hfill \\
	 Initialisierung der Semaphore
\item[semop] \hfill \\
	Manipulieren der Semaphore
\end{description}

\begin{lstlisting}
int max = 30;
MyType a[max];
unsigned int ni = 0;	// number of items
Semaphore mutex(1);
Sempahore aItems[0]; 	// available items
Semaphore aPlaces[max];	// available places

// Producer
aPlaces.down(1);
mutex.down(1);
//critical section ...
a[ni] = item; // bei structs / strings verwendet man memcpy
ni = ni + 1;
mutex.up(1);
aItems.up(1);

// Consumer
aItem.down(1);
mutex.down(1);
//critical section;
myType myItem = a[ni-1];
ni = ni - 1;
mutex.up(1);
aPlaces.up(1);
\end{lstlisting}

    
\subsection{Wartegraphen (Ziel: Deadlock-Erkennung)}

\subsubsection{Deadlock-Bedingung}
Alle Prozesse liegen in einer starken Zusammenhangskomponente des Wartegraphen, aus der kein Weg hinausführt, bzw. Wege hinaus wieder in starke Zusammenhangskomponenten führen.

\subsubsection{Welche Prozesse sind im Deadlock?}
Alle Prozesse, die in der gegenwärtigen (vom Deadlock betroffenen) SCC enthalten sind und alle die in SCC's liegen, die von diesem abhängen.

\subsubsection{Deadlockvermeidung oder Auflöusng}
\begin{itemize}
\item[Ignorieren] Wenn ein Deadlock selten auftritt und eine Lösung aufwändig wäre, könnte man ihn unter Umständen ignorieren, wenn das Kosten-Nutzen Verhältnis nicht dagegen spricht.

\item[Betriebsmittelgraphen-Analyse] Untersuchung des Betriebsmittelgraphen (Tarjan) und Aufheben des Deadlocks durch Zwangsentfernen eines Prozesses.

\item[Exklusivität] Spooling

\item[Alles auf einmal angefordern] Man könnte alle Ressourcen auf einmal anfordern, was aber eine große Ressourcenverschwendung darstellt

\item[Zyklus verhindern] Ressourcen nur nach aufsteigender Nummer anfordern und belegen.

\item[Bankiersalgorithmus] Maximale Anforderung der Betriebsmittel und ihre Verfügbarkeit müssen im voraus bekannt sein.
\end{itemize}

\subsubsection{Deadlock-Erkennung: Starke Zusammenhangskomponenten (Tarjan)}
\begin{verbatim}
Eingabe: Graph G = (V, E)

maxdfs := 0                      // Zähler für dfs
U := V                           // Menge der unbesuchten Knoten
S := {}                          // Stack zu Beginn leer
while (es gibt ein v in U) do   // Solange es bis jetzt unerreichbare Knoten gibt
  tarjan(v)                     // Aufruf arbeitet alle von v0 erreichbaren Knoten ab
end while
\end{verbatim}

\begin{verbatim}
procedure tarjan(v)
v.dfs := maxdfs;           // Tiefensuchindex setzen
v.lowlink := maxdfs;       // v.lowlink <= v.dfs
maxdfs := maxdfs + 1;      // Zähler erhöhen
S.push(v);                 // v auf Stack setzen
U := U \ {v};              // v aus U entfernen
forall (v, v') in E do     // benachbarte Knoten betrachten
  if (v' in U)
    tarjan(v');            // rekursiver Aufruf
    v.lowlink := min(v.lowlink, v'.lowlink);
  // Abfragen, ob v' im Stack ist. 
  // Bei geschickter Realisierung in O(1).
  // (z.B. Setzen eines Bits beim Knoten beim "push" und "pop") 
  elseif (v' in S)
    v.lowlink := min(v.lowlink, v'.dfs);
  end if
end for
if (v.lowlink = v.dfs)     // Wurzel einer SZK
  print "SZK:";
  repeat
    v' := S.pop;
    print v';
  until (v' = v);
end if
\end{verbatim}

\subsection{shared memory}

\begin{description}
\item [shmget] \hfill \\ Erstellt oder bezieht vorhandenen gemeinsamen Speicher über eine Key. (liefert die segment id zurück)
\item [shmat] \hfill \\ Bindet gemeinsamen Speicher ein (bildet shared memory im lokalen addressraum ab)
\item [shmdt] \hfill \\ Entfernt (detach) den gemeinsamen Speicher
\item [shmctl] \hfill \\ Kann z.B. dazu dienen shared memory komplett zu entfernen.
\end{description}

\subsection{pthreads}



\section{Netzwerke}




\section{Scheduling}




\section{Memory Management}

\subsection*{Nicht Virtuelle Speicherverwaltung}

        direkter Zugriff auf Speicher
        Swapping
        Fragmentierung
        Speicheraddressen stehen nicht zu Compilezeit fest
        static allocation / dynamic allocation
     (mit virtuellem Memory Management)



\subsection*{Virtuelle Speicherverwaltung}

        Abbildung von virtuellen auf physikalischen Adressen
        Aufteilung des virtuellen Addressraums in Pages
        Jede Page hat eigene Pagintabelle
        Hauptspeicher ist unterteilt in Pageframes gleicher Größe
        Hardwareunterstützung durch MMU
        Translation lockaside buffer (http://de.wikipedia.org/wiki/Translation\_Lookaside\_Buffer)
        Pagefault
        Segmentation fault





\section{Sicherheit}

    Subjekt (Aktiv bei Zugriff auf Objekte)
    Objekt (Was wir schützen möchten)
    Subjekte können auch Objekte sein
    ACL (Access Control Lists)    

\subsection{Subjekt}

\subsection{Objekt}

\subsection{ACL (Access Control Lists)}
  




% This included Fragment comprises of questions that revolve multithreading
%\section{Nebenläufigkeit}

    
\subsection{Race conditions}
Ein \textbf{race condition} ist in der Programmierung eine Konstellation, in der das \crucial{Ergebnis einer Operation vom zeitlichen Verhalten (bzw. Ausf"uhrungsreihenfolge) bestimmter Einzeloperationen abh"angt}. Das Ergebnis wäre also nichtdeterministisch.

\subsection{Synchronisation}
Die Synchronisation dient dem \crucial{Zweck} \crucial{racing conditions} zu \crucial{vermeiden}. Das \crucial{Ziel} der Synchronisation ist die \crucial{Serialisierbarkeit!}

\subsection{Serialisierbarkeit}
\crucial{Reihenfolge bei konkurrierendem Zugriff entspricht seriellem Zugriff}.

\paragraph*{}
Beispiel: Prozesse $P_1, ..., P_n$ operieren auf eine Datenstruktur $d$. Die konkurrierende Operationen auf $d$ heißt serialisierbar, wenn es eine sequentielle Ausführung $P_{i_{1}}, P_{i_{2}}, ..., P_{i_{n}}$ gibt, die $d$ im selben Nachzustand hinterlässt.

\subsection{Critical Section}
\crucial{Code-Abschnitte in Prozesse $P_1, ..., P_2$ deren nebenl"aufige Ausf"uhrung die Serialisierbarkeit verletzen w"urde}.

\subsection{Mutual Exclusion}
\crucial{Die Ausf"uhrung im critical section schlie"st sich Wechselseitig aus}.

\subsection{Peterson's Algorithmus}

Aufruf durch einen Prozess:
\begin{lstlisting}
enter(pid); // pid is 0 or 1
// process CS...
leave(pid);
\end{lstlisting}


\begin{lstlisting}
// global
int turn; // 0 or 1
int interested[] = {0, 0}

void enter(int pid) { // pid is either 0 or 1
	interested[pid] = 1;
	turn = pid;
	while (turn == pid && interested[1-pid]); // active wait
}

void leave(int pid) {
	interested[pid] = 0;
}
\end{lstlisting}


\subsection{Strict Alternation}
Wandelt man Peterson's Algorithmus leicht ab, indem man sich der \texttt{interested} variable entledigt ist das Resultat \textbf{strict alternation}. Dabei wartet ein Prozess so lange, bis ein anderer Prozess versucht in die critical section einzutreten. Erst dann kann er die kritische Sektion betreten.

\begin{lstlisting}
// global
int turn; // 0 or 1

void enter_critical(int pid) { // pid is either 0 or 1
	turn = pid;
	while (turn == pid); // active wait
}
\end{lstlisting}


\subsection{Fischers Protokoll}
\begin{lstlisting}
// globale Variablen 
volatile int turn = 0; 
void enterCritical(int pid) {	// pid >= 1 	
	do { 		
		while(turn != 0); 		
		turn = pid; 	
	} while(turn != pid); 
} 

void leaveCritical(int pid) { 	turn = 0; } 
\end{lstlisting}

Nachteil: Ein Prozess könnte theoretisch verhungern.

\subsection{Verhungerungsfreiheit (Free of Starvation)}
Wenn ein Prozess kontinuierlich bereit ist, bekommt er schließlich die Ressource (critical section).

\subsection{Spinlocks (Active Wait)}
Aktives Warten (spinlock) ist das \crucial{Verharren} eines Prozesses \crucial{in einer Schleife, bis} der zu betretende \crucial{kritische Abschnitt} wieder \crucial{frei ist}.


% A table may be more appropriate
\begin{description}
\item[Vorteil:] \hfill \\
	Keine Betriebssystem-Beteiligung (kein teurer Kontextwechsel erforderlich)
\item[Nachteil:] \hfill \\
	Wartende Prozesse verbrauchen Rechenzeit
\end{description}

\begin{verbatim}
    Gemeinsam von A und B genutzte Variable: lock
    Interpretation des Werts:                0 gesperrt, ungleich 0 offen 
    Initialisierung:                         lock = 0
           
    Prozess A                    Prozess B
       ...                          ...
       solange (lock == 0) {        ...
          ;                         lock = 1; Aktion b
       }                            ...
       Aktion a                     ...
\end{verbatim}

\subsection{Prioriäten Inversion}
Szenario: P1 (gute Priorität), P2 (schlechte Priorität) werden auf einem core ausgeführt. P2 kommt in die CS und P1 muss aktiv warten.

\paragraph*{Wirkung:}
P1 bekommt mehr Rechenzeit, die aber in AW (aktive wait) vergeudet wird, da P2 vom Scheduler aufgrund seiner schlechten Priorität immer wieder unterbrochen wird und die CS noch nicht verlassen hat.

\subsection{Lifelock}
\crucial{Keine Nutzdatenverarbeitung obwohl sich der Programmcounter "andert}. Im gegensatz dazu stehen bei einem Deadlock alle Programmcounter fest.

\subsection{Kooperatives Multitasking und präemptives Scheduling}

\begin{description}
\item [Kooperatives Multitasking] \hfill \\
	Prozesswechsel erfolgt \ul{nur}, wenn der aktive Prozess \ul{freiwillig} die CPU wieder abgibt. (sogenanntes nicht-präemptives Scheduling)
\item [Präemptives Scheduling] \hfill \\
	Pr"aemptive Scheduler k"onnen einem aktiven Prozess die CPU \uline{entziehen} (Technik: Clockinterrupt führt zum Scheduler-Aufruf).
\end{description}


\subsection{Reader / Writer (nonblocking write protocol)}
Annahme: 1 leser (liest den Datenblock) und 1 schreiber (beschreibt den Datenblock) \\
Ziel: Der Reader verarbeitet nur konsistene Daten, bei denen er beim Lesen nicht durch den Writer unterbrochen wurde.

\subsubsection{Optimistisches Verfahren}
\begin{enumerate}
\item Writer darf immer schreiben
\item Reader liest \crucial{optimistisch}, prüft dann , ob die Daten konsistent sind
\end{enumerate}

\begin{lstlisting}
// global variables
int ccf = 0;
\end{lstlisting}

\begin{lstlisting}
// writer
ccf = ccf + 1;
// write data
ccf = ccf + 1; 
\end{lstlisting}

\begin{lstlisting}
// reader
do {
	int ccfbegin, ccfend;
	do {
		ccfbegin = ccf;
	} while (ccfbegin % 2);
	// read data
	ccfend = ccf;
} while (ccfbegin != ccfend);
\end{lstlisting}


\subsection{Ringpuffer}

\subsubsection{Eigenschaften}
\begin{itemize}
\item Reader-Writer Kommunikation mit einem Leser und einem Schreiber
\item FIFO-Buffer
\item Ohne Betriebssystem-Beteiligung
\item Ungelesene Daten dürfen \uline{noch} nicht überschrieben werden
\item ungeschriebene Daten dürfen \uline{noch} nicht gelesen werden
\item Puffer ist leer, wenn \texttt{rIdx == wIdx}
\item Puffer ist voll, wenn \verb|(wIdx + 1) % n == rIdx|
\item Bedingung fürs Lesen \texttt{(rIdx != wIdx)}
\item Bedingung fürs Schreiben \verb|(wIdx + 1) % n != rIdx|
\item Bei einer Arraygröße von $n$ ergibt sich eine Kapazität von $n -1$
\item \verb|%| (Modulo) Operation ist unter umst"anden teuer. W"ahle $n=2^k$ mit 
$k \in \mathbb{N}$ und ersetze \verb|%n| durch \verb|&(1<<k)-1|
\end{itemize}

\begin{lstlisting}
void write(t in) {
	unsigned nextWIdx = (wIdx + 1) % n;
	while (nextWIdx == rIdx); // spinlock
	buffer[wIdx] = in; // write
	wIdx = nextWIdx; // update write index
}

t read() {
	t out;
	while (rIdx == wIdx); // spinlock
	out = buffer[rIdx]; 
	rIdx = (rIdx + 1) % n;
	return out;
}
\end{lstlisting}

\subsection{Implementierung des Consumer Producer Konzeptes mit Semaphoren}

\subsubsection{Semaphoren}

Semaphoren dienen der Sicherung von kritischen Abschnitten. Sie verfügbt über zwei grundlegende Funktionen:

\begin{enumerate}
\item lock anfordern
\item lock freigeben
\end{enumerate}

\begin{itemize}
\item lock wird angefordert
\item wenn der lock nicht verfügbar ist, wird der Prozess schlafen gelegt (aus der Liste der aktiven Prozesse (runqueue) entfernt)
\item wenn lock bezogen wurde
\end{itemize}

\begin{description}
\item[semget] \hfill \\
	 Semaphore beantragen
\item[semctl] \hfill \\
	 Initialisierung der Semaphore
\item[semop] \hfill \\
	Manipulieren der Semaphore
\end{description}

\begin{lstlisting}
int max = 30;
MyType a[max];
unsigned int ni = 0;	// number of items
Semaphore mutex(1);
Sempahore aItems[0]; 	// available items
Semaphore aPlaces[max];	// available places

// Producer
aPlaces.down(1);
mutex.down(1);
//critical section ...
a[ni] = item; // bei structs / strings verwendet man memcpy
ni = ni + 1;
mutex.up(1);
aItems.up(1);

// Consumer
aItem.down(1);
mutex.down(1);
//critical section;
myType myItem = a[ni-1];
ni = ni - 1;
mutex.up(1);
aPlaces.up(1);
\end{lstlisting}

    
\subsection{Wartegraphen (Ziel: Deadlock-Erkennung)}

\subsubsection{Deadlock-Bedingung}
Alle Prozesse liegen in einer starken Zusammenhangskomponente des Wartegraphen, aus der kein Weg hinausführt, bzw. Wege hinaus wieder in starke Zusammenhangskomponenten führen.

\subsubsection{Welche Prozesse sind im Deadlock?}
Alle Prozesse, die in der gegenwärtigen (vom Deadlock betroffenen) SCC enthalten sind und alle die in SCC's liegen, die von diesem abhängen.

\subsubsection{Deadlockvermeidung oder Auflöusng}
\begin{itemize}
\item[Ignorieren] Wenn ein Deadlock selten auftritt und eine Lösung aufwändig wäre, könnte man ihn unter Umständen ignorieren, wenn das Kosten-Nutzen Verhältnis nicht dagegen spricht.

\item[Betriebsmittelgraphen-Analyse] Untersuchung des Betriebsmittelgraphen (Tarjan) und Aufheben des Deadlocks durch Zwangsentfernen eines Prozesses.

\item[Exklusivität] Spooling

\item[Alles auf einmal angefordern] Man könnte alle Ressourcen auf einmal anfordern, was aber eine große Ressourcenverschwendung darstellt

\item[Zyklus verhindern] Ressourcen nur nach aufsteigender Nummer anfordern und belegen.

\item[Bankiersalgorithmus] Maximale Anforderung der Betriebsmittel und ihre Verfügbarkeit müssen im voraus bekannt sein.
\end{itemize}

\subsubsection{Deadlock-Erkennung: Starke Zusammenhangskomponenten (Tarjan)}
\begin{verbatim}
Eingabe: Graph G = (V, E)

maxdfs := 0                      // Zähler für dfs
U := V                           // Menge der unbesuchten Knoten
S := {}                          // Stack zu Beginn leer
while (es gibt ein v in U) do   // Solange es bis jetzt unerreichbare Knoten gibt
  tarjan(v)                     // Aufruf arbeitet alle von v0 erreichbaren Knoten ab
end while
\end{verbatim}

\begin{verbatim}
procedure tarjan(v)
v.dfs := maxdfs;           // Tiefensuchindex setzen
v.lowlink := maxdfs;       // v.lowlink <= v.dfs
maxdfs := maxdfs + 1;      // Zähler erhöhen
S.push(v);                 // v auf Stack setzen
U := U \ {v};              // v aus U entfernen
forall (v, v') in E do     // benachbarte Knoten betrachten
  if (v' in U)
    tarjan(v');            // rekursiver Aufruf
    v.lowlink := min(v.lowlink, v'.lowlink);
  // Abfragen, ob v' im Stack ist. 
  // Bei geschickter Realisierung in O(1).
  // (z.B. Setzen eines Bits beim Knoten beim "push" und "pop") 
  elseif (v' in S)
    v.lowlink := min(v.lowlink, v'.dfs);
  end if
end for
if (v.lowlink = v.dfs)     // Wurzel einer SZK
  print "SZK:";
  repeat
    v' := S.pop;
    print v';
  until (v' = v);
end if
\end{verbatim}

\subsection{shared memory}

\begin{description}
\item [shmget] \hfill \\ Erstellt oder bezieht vorhandenen gemeinsamen Speicher über eine Key. (liefert die segment id zurück)
\item [shmat] \hfill \\ Bindet gemeinsamen Speicher ein (bildet shared memory im lokalen addressraum ab)
\item [shmdt] \hfill \\ Entfernt (detach) den gemeinsamen Speicher
\item [shmctl] \hfill \\ Kann z.B. dazu dienen shared memory komplett zu entfernen.
\end{description}

\subsection{pthreads}



% This includes network specific subjects
%\include{fragment_netzwerke}

% This includes security specific subjects
%\section{Sicherheit}

    Subjekt (Aktiv bei Zugriff auf Objekte)
    Objekt (Was wir schützen möchten)
    Subjekte können auch Objekte sein
    ACL (Access Control Lists)    

\subsection{Subjekt}

\subsection{Objekt}

\subsection{ACL (Access Control Lists)}
  



\end{document}

