\section{Netzwerke}

\subsection{Socket}
Eine bidirektionale Software-Schnittstelle zur Interprozess- (IPC) oder Netzwerk-Kommunikation Ein Kommunikationsendpunkt.

\begin{description}
\item[Stream Socket (z.B. TCP/IP)] \hfill \\
Streamorientiert. In Bezug auf Inhalt und Reihenfolge gesichert (Punkt zu Punkt). Partielle Ordnung (Datenpakete sind in Bezug auf ein und denselben Sender wohlgeordnet.)
\item[Datagram Socket (z.B. UDP/IP)] \hfill \\
Datagramm-/Telegrammorientiert. Ungesichert: Verlust von Datagrammen und Vertauschung von Datagrammen. (Ein Telegramm ist nur korrekt, wenn sie zeitlich und inhaltlich korrekt ist).
\end{description}


\subsection{Client / Server}
Initiator ist nur der Client. Server antwortet nur, wenn er gefragt wird.

\subsubsection{Client}
\begin{lstlisting}
s = socket (...); // Socket erstellen
connect (s, ...); // Mit Socket verbinden
send (s, ...); // senden! alternativ write, bei UDP sendto()
recv (s, ...); // empfangen! alternative read. bei UDP recvfrom()
close (s, ...); // Socket schliessen
\end{lstlisting}

\subsubsection{Server}
\begin{lstlisting}
s = socket (...); // Socket erstellen
bind (...); Binde Port Adresse
listen (...); Warte auf Anfragen
while (1) {
	s2 = accept (...); // Verbindung annehmen
	recv (...);
	send (...);
	close (s2);
}
send(s, ...); // senden! alternativ write, bei UDP sendto()
recv(s, ...); // empfangen! alternative read. bei UDP recvfrom()
close(s, ...); // Socket schliessen
\end{lstlisting}

\subsection{OSI Schichten Model}
ein Referenzmodell für Netzwerkprotokolle als Schichtenarchitektur. Die Schichten sind wie folgt:

\begin{enumerate}
\item[Schicht 7] – Anwendungsschicht
\item Schicht 6] – Darstellungsschicht
\item Schicht 5] – Sitzungsschicht
\item[Schicht 4] – Transportschicht
\item[Schicht 3] – Vermittlungsschicht
\item[Schicht 2] – Sicherungsschicht
\item[Schicht 1] – Bit"ubertragungsschicht

\end{enumerate}

\subsection{Port}
Ein Port ist der Teil einer Netzwerk-Adresse, der die Zuordnung von TCP- und UDP-Verbindungen und -Datenpaketen zu Server- und Client-Programmen durch Betriebssysteme bewirkt. (Quelle: Wikipedia).

Ports dienen zwei Zwecken:
\begin{itemize}
\item Primär sind Ports ein Merkmal zur Unterscheidung mehrerer Verbindungen zwischen demselben Paar von Endpunkten.[1]
\item Ports können auch Netzwerkprotokolle und entsprechende Netzwerkdienste identifizieren.
\end{itemize}

\subsection{Kommunikationsarten}

\begin{description}
\item[Unidirektional] \hfill \\
\item[Halbduplex] \hfill \\
\item[Vollduplex] \hfill \\
\item[Broadcast] \hfill \\
\item[Multicast] \hfill \\
\item[Point to Point] \hfill \\
\end{description}

\subsection{sigpipe}


\subsection{Wiederverwendung von Ports}

\subsection{Ethernet packet und ethernet frame}

\subsection{MTU}
Die Maximum Transmission Unit (MTU) beschreibt die maximale Paketgröße eines Protokolls der Vermittlungsschicht (Schicht 3) des OSI-Modells, welche ohne Fragmentierung in den Rahmen (engl. "Frame") eines Netzes der Sicherungsschicht (Schicht 2) übertragen werden kann. (Wikipedia: http://de.wikipedia.org/wiki/Maximum\_Transmission\_Unit)

\subsection{packete defragmentieren}
MTU Einstellung ...

\subsection{IP Header}

\subsection{UDP Header}

\subsection{select / poll}